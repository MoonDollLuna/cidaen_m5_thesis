% -------------------------------
% CIDaeN: Plantilla para la elaboración del TFM. Versión 0.0 
%
% Luis de la Ossa. UCLM
% 
% Compilar con XeLaTeX 
% -------------------------------



% -------------------------------
% Dedicatoria
% -------------------------------
\cleardoublepage
\thispagestyle{empty}

\vspace*{9cm}  
\begin{flushright} \em 
Dedicado a la gente que, pese a todo,\\
sigue persiguiendo sus sueños.\\
Nunca os rindáis.
\end{flushright}

% -------------------------------
% Declaración de autoría
% -------------------------------

\cleardoublepage
\thispagestyle{plain}
\setcounter{page}{1} \null
\begin{center}
\Large{\bft{Declaración de autoría}}
\end{center}
\vskip1cm

Yo, \textbf{Luna Jiménez Fernández}, con DNI \textbf{47092045M}, declaro que soy la única autora del Trabajo Fin de Master titulado \textbf{\textit{``WiDS Dathaton 2024 - Challenge 2: Modelos de regresión para estimación del periodo de diagnóstico metastático''}}, que el citado trabajo no infringe las leyes en vigor sobre propiedad intelectual, y que todo el material no original contenido en dicho trabajo está apropiadamente atribuido a sus legítimos autores.

\vspace*{2cm}
\begin{center}
Albacete, a \textbf{9} de \textbf{Junio} de \textbf{2025}

\vskip3cm

Fdo.: \textbf{\autor}
\end{center}


% -------------------------------
% Resumen
% -------------------------------
\cleardoublepage
\thispagestyle{plain}
\begin{center}
\Large{\bft{Resumen}}
\end{center}
\vskip1cm

El acceso equitativo a una \textbf{atención sanitaria de calidad} es un problema de interés a nivel global - existiendo diferencias significativas tanto en \textbf{calidad} como en \textbf{posibilidad} de acceso entre poblaciones. Este problema afecta especialmente al \textbf{cáncer de mama triple negativo} y sus metástasis - unos de los cánceres más difíciles de tratar, y en los que más pueden influir dichos sesgos retrasando su diagnóstico y tratamiento.

Por tanto, el objetivo de este trabajo consiste en \textbf{desarrollar un modelo de regresión} capaz de predecir el tiempo de diagnóstico de metástasis - incidiendo a la vez en la relevancia de los factores \textbf{geográficos, socioeconómicos y climáticos} - a través de un proceso de \textbf{ciencia de datos}.

Se ha realizado una \textbf{revisión de las principales técnicas} utilizadas durante el trabajo: el ciclo de vida de la ciencia de datos y algunos de los principales modelos de aprendizaje automático de regresión, con especial foco en los \textit{ensembles} de \textit{Gradient Boosting}.

Además, se ha realizado un \textbf{análisis exploratorio del conjunto de datos}, observando la gran influencia de los códigos de diagnóstico a la vez que la irrelevancia de los atributos numéricos - representativos de factores geográficos, socioeconómicos y climáticos. Esta información se ha podido utilizar para realizar un \textbf{pre-procesamiento} de los datos, proponiendo tres subconjuntos de atributos a estudiar y varias transformaciones para imputar valores perdidos.

Para el desarrollo del modelo, se ha realizado una fase de \textbf{modelado} en la que se han propuesto 16 modelos de regresión - realizando sobre éstos un ajuste de hiperparámetros y una selección de modelo en base al error. Además, el modelo seleccionado ha sido \textbf{evaluado} - estudiando su rendimiento tanto durante las fases de ajuste de hiperparámetros como de selección de modelos, y contrastando su rendimiento con un equivalente, ganador de la competición en la que se ha basado este problema.

Finalmente, se ha \textbf{desplegado} el modelo a través de una aplicación web para permitir su uso de forma pública.

% -------------------------------
% Abstract
% -------------------------------
\cleardoublepage
\thispagestyle{plain}
\begin{center}
	\Large{\bft{Abstract}}
\end{center}
\vskip1cm

Equitative access to \textbf{quality healthcare} is a problem of global interest - with significant differences in both \textbf{quality} and \textbf{availability} existing depending on the population. This is especially true for \textbf{triple negative breast cancer} and its metastasis - one of the most dangerous and difficult to treat cancers, and one in which negative biases can greatly influence by delaying its diagnosis and treatment.

The goal of this work is \textbf{developing a regression model} capable of predicting the metastasis diagnosis period of a patient - with a special focus on studying the relevance of \textbf{geographic, socioeconomic and climatic} factors - via data science.

A survey of the \textbf{state of the art} of the main techniques used during this workk has been performed: studying the data science life cycle and some of the main regression machine learning models, focusing on \textit{Gradient Boosting ensembles}.

In addition, an \textbf{exploratory data analysis} has been done on the dataset, finding that diagnosis codes carry a significant influence on the diagnosis period, while most numerical attributes - representing geographic, socioeconomic and climatic attributes - are mostly unrelated to the prediction. This knowledge has been used to \textbf{pre-process} the data, proposing three feature subsets to study and several transformations, including missing value imputing.

For the model development, a \textbf{modelling} phase has been done, in which 16 regression models have been proposed - performing hyperparameter optimization and model selection on them based on their error. In addition, the chosen model has been \textbf{evaluated} - studying its performance in both previous phases, and comparing it with a similar model that placed first on the competition that originally proposed this problem.

Finally, a web application has been \textbf{deployed} - to allow users to access the developed model publically.
% -------------------------------
% Agradecimientos
% -------------------------------
\cleardoublepage
\thispagestyle{plain}
\begin{center}
\Large{\bft{Agradecimientos}}
\end{center}
\vskip1cm

En primer lugar, quiero agradecer a todos mis compañeros y amigos del grupo de \textbf{Sistemas Informáticos y Minería de Datos (SIMD)} - y, especialmente, a mi amigo y director \textbf{Juan Carlos Alfaro Jiménez} - por su apoyo, recursos y consejos durante la realización de este trabajo. Aunque ya no sea formalmente parte de este grupo, siempre me sentiré vinculada a él.\\

Además, quiero a gradecer a mis amigos y familia del \textbf{Curso de Comic Online de la Escola Joso} - \textbf{Arai, Aina, Arkaitz, Clara, Irene, Martín, Pau, Rafi...} -, con los que compartí un proyecto de gran importancia personal, mi primer comic publicado, y en los que he encontrado un grupo al que pertenecer. Muchas gracias por todo.\\

Finalmente, quiero agradecer a \textbf{mi familia y seres queridos} - tanto los que me acompañan presencialmente como los que se encuentran a distancia. Vuestro apoyo y cariño continuo me ha ayudado a seguir adelante y acabar este trabajo a pesar de todas las dificultades.