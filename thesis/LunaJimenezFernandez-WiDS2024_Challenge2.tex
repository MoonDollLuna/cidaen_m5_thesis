% -------------------------------
% CIDaeN: Plantilla para la elaboración del TFM. Versión 0.0 
%
% Luis de la Ossa. UCLM
% 
% Compilar con XeLaTeX 
% -------------------------------
\documentclass[12pt, a4paper]{book}

% -------------------------------
% Información y opciones del documento. Debe editarse.
% -------------------------------
% -------------------------------
% CIDaeN: Plantilla para la elaboración del TFM. Versión 0.0 
%
% Luis de la Ossa. UCLM
% 
% Compilar con XeLaTeX 
% -------------------------------


% -------------------------------
% Comandos comunes
% -------------------------------
\newcommand{\uclm}{Universidad de Castilla-La Mancha\,}
\newcommand{\dsi}{Departamento de Sistemas Informáticos\,}
\newcommand{\esii}{Escuela Superior de Ingeniería Informática de Albacete\,}



% -------------------------------
% Comandos específicos
% -------------------------------
\newcommand{\autor}{Luna Jiménez Fernández\,}
\newcommand{\titulo}{WiDS Dathaton 2024 - Challenge 2:\\
	Modelos de regresión para estimación del periodo de diagnóstico metastático\,}
\newcommand{\director}{Juan Carlos Alfaro Jiménez\,}
\newcommand{\codirector}{\,}
\newcommand{\fecha}{Junio, 2025\,}








% -------------------------------
% Configuración del documento, paquetes, etc. 
% -------------------------------
\input{include/configuracion}

%---------------------------------
% Bibliografia
%----------------------------------
\usepackage[style=ieee]{biblatex}
\addbibresource{bibliography.bib}

%------------------------------
% Espacio entre parrafos
%------------------------------
% \usepackage{parskip}
\usepackage{float}

% Documento
\begin{document}

% -------------------------------
% Portada y título
% -------------------------------
% Fuente  de la portada
\setmainfont[Ligatures={NoRequired,NoCommon,NoContextual}]{Calibri}
\input{elements/portada}

% -------------------------------
% Fuente del documento.
% -------------------------------
\setmainfont[Ligatures={NoRequired,NoCommon,NoContextual}]{Calibri}

% -------------------------------
% Preámbulo
% -------------------------------
\frontmatter 
% -------------------------------
% CIDaeN: Plantilla para la elaboración del TFM. Versión 0.0 
%
% Luis de la Ossa. UCLM
% 
% Compilar con XeLaTeX 
% -------------------------------



% -------------------------------
% Dedicatoria
% -------------------------------
\cleardoublepage
\thispagestyle{empty}

\vspace*{9cm}  
\begin{flushright} \em 
Dedicado a la gente que, pese a todo,\\
sigue persiguiendo sus sueños.\\
Nunca os rindáis.
\end{flushright}

% -------------------------------
% Declaración de autoría
% -------------------------------

\cleardoublepage
\thispagestyle{plain}
\setcounter{page}{1} \null
\begin{center}
\Large{\bft{Declaración de autoría}}
\end{center}
\vskip1cm

Yo, \textbf{Luna Jiménez Fernández}, con DNI \textbf{47092045M}, declaro que soy la única autora del Trabajo Fin de Master titulado \textbf{\textit{``WiDS Dathaton 2024 - Challenge 2: Modelos de regresión para estimación del periodo de diagnóstico metastático''}}, que el citado trabajo no infringe las leyes en vigor sobre propiedad intelectual, y que todo el material no original contenido en dicho trabajo está apropiadamente atribuido a sus legítimos autores.

\vspace*{2cm}
\begin{center}
Albacete, a \quad \ldots \quad de \quad \ldots \quad de 2025

\vskip3cm

Fdo.: \textbf{\autor}
\end{center}


% -------------------------------
% Resumen
% -------------------------------
\cleardoublepage
\thispagestyle{plain}
\begin{center}
\Large{\bft{Resumen}}
\end{center}
\vskip1cm

TODO RESUMEN AQUI

% -------------------------------
% Abstract
% -------------------------------
\cleardoublepage
\thispagestyle{plain}
\begin{center}
	\Large{\bft{Abstract}}
\end{center}
\vskip1cm

TODO ABSTRACT HERE

% -------------------------------
% Agradecimientos
% -------------------------------
\cleardoublepage
\thispagestyle{plain}
\begin{center}
\Large{\bft{Agradecimientos}}
\end{center}
\vskip1cm

En primer lugar, quiero agradecer a todos mis compañeros y amigos del grupo de \textbf{Sistemas Informáticos y Minería de Datos (SIMD)} - y, especialmente, a mi amigo y director \textbf{Juan Carlos Alfaro Jiménez} - por su apoyo, recursos y consejos durante la realización de este trabajo. Aunque ya no sea formalmente parte de este grupo, siempre me sentiré vinculada a él.\\

Además, quiero a gradecer a mis amigos y familia del \textbf{Curso de Comic Online de la Escola Joso} - \textbf{Arai, Aina, Arkaitz, Clara, Irene, Martín, Pau, Rafi...} -, con los que compartí un proyecto de gran importancia personal, mi primer comic publicado, y en los que he encontrado un grupo al que pertenecer. Muchas gracias por todo.\\

Finalmente, quiero agradecer a \textbf{mi familia y seres queridos} - tanto los que me acompañan presencialmente como los que se encuentran a distancia. Vuestro apoyo y cariño continuo me ha ayudado a seguir adelante y acabar este trabajo a pesar de todas las dificultades.

% -------------------------------
% Índices y tablas
% -------------------------------
% EXTEND THE TOC TO SHOW SECTIONS
% \setcounter{docdepth}{1}

\tableofcontents	% Índice
\listoffigures		% Índice de figuras
\listoftables		% Índice de tablas
\listofalgorithms   % Índice de algoritmos (comentar si no los hay)
\listofcodes	    % Índice de códigos (comentar si no los hay)
\cleardoublepage

% -------------------------------
% Documento
% -------------------------------
\mainmatter
% Estilo de página
\pagestyle{fancy}



% CAPITULOS
% 1 - Introducción y descripción del problema
\chapter{Introducción}

\section{Objetivos}

\section{Estructura de la memoria}

% 2 - SOTA de los modelos utilizados
\chapter{Revisión de técnicas}

\section{Ciencia de datos}

\subsection{Búsqueda de hiperparámetros y validación cruzada}

\section{Modelos propuestos}

\subsection{Modelos lineales}

\subsection{Máquinas de vectores de soporte}

\subsection{Árboles de decisión y \textit{ensembles}}

% 3 - EDA del conjunto de datos
\chapter{Estudio del problema}

\section{Definición del problema}

\subsection{Atributos del problema}

\section{Análisis exploratorio de datos}

\subsection{Variable objetivo - distribución y comportamiento}

\subsection{Valores perdidos}

\subsection{Atributos categóricos}

\subsection{Atributos numéricos}

\subsection{Variables geográficas, sociales y económicas} 

% 4 - Preprocesamiento de los datos
\chapter{Preparación del conjunto de datos}

\section{Selección de atributos}

\subsection{Selección manual de atributos}

\subsection{Selección automática de atributos}

\section{Preprocesamiento de los datos}

% 5 - Modelado y experimentación
\chapter{Modelado y evaluación}

\section{Selección de modelos}

\section{Experimentación}

\subsection{Ajuste de hiperparámetros y selección de subconjuntos de atributos}

\subsection{Validación y selección de modelo final}

\section{Análisis de resultados}

\subsection{Rendimiento de los subconjuntos de hiperparámetros}

\subsection{Rendimiento de los modelos entrenados}

\subsection{Rendimiento del modelo final}

% 6 - Aplicación web
\chapter{Aplicación web}

\section{Aplicación para usuario - predicción individual}

\section{Aplicación \textit{batch} - predicción en grupo}

% 8 - Conclusiones
\chapter{Conclusiones}

A lo largo del trabajo realizado y de la memoria, se ha descrito en detalle el \textbf{ciclo de vida de la ciencia de datos} llevado a cabo para resolver el problema planteado: la \textbf{predicción del tiempo de diagnóstico de cáncer metastásico}, en base a datos médicos, geográficos, socioeconómicos y climáticos.

El primer paso ha sido un \textbf{análisis exploratorio de datos}, para conocer y entender mejor el comportamiento del conjunto de datos - lo que ha permitido realizar un \textbf{preprocesamiento} posterior en el que se han seleccionando atributos y transformando los datos para su uso durante el \textbf{modelado}, donde se han planteado una serie de posibles modelos para solucionar el problema. 

El mejor de dichos modelos ha sido seleccionado a través de una \textbf{evaluación} de su rendimiento en comparación con el resto de modelos y con los ganadores de la competición. Finalmente, este modelo se ha dejado disponible a los usuarios finales a través del \textbf{despliegue} de una aplicación web.

Tras el trabajo realizado, se puede afirmar que el \textbf{primer objetivo} - la creación de un modelo de regresión capaz de predecir el tiempo de diagnóstico - se ha \textbf{cumplido con éxito} dentro de las posibilidades:
\begin{itemize}
	\item El modelo entrenado \textbf{presenta un error elevado} - alrededor de \textbf{82 días} en promedio -, si bien dicho error se encuentra cercano al error de los modelos ganadores de la competición.
	\item El funcionamiento del modelo es \textbf{ágil} - en el orden de los segundos para el entrenamiento y de los milisegundos para la predicción -, lo que lo hace \textbf{utilizable en tiempo real}.
	\item La \textbf{aplicación web} permite el uso simple del modelo por parte de los usuarios finales.
\end{itemize}

Respecto al \textbf{segundo objetivo} - el estudio de la influencia de factores geográficos, socioeconómicos y climáticos -, se han extraído las siguientes conclusiones:
\begin{itemize}
	\item El análisis exploratorio de datos indica que \textbf{solo la información médica es relevante para la predicción del tiempo} - los atributos socioeconómicos y climáticos \textbf{no presentan correlación clara con el tiempo de diagnóstico}, y el único atributo geográfico con cierta relevancia es el \textbf{estado de residencia del paciente}.
	\item Pese a esto, \textbf{el modelo final seleccionado incluye información socioeconómica} - en la forma de información estadística sobre los estudios universitarios y los ingresos de las familias. Ahora bien, en la práctica \textbf{dichas variables no tienen apenas influencia sobre el tiempo predicho}.
\end{itemize}

Actualmente, el trabajo completo se encuentra disponible en un repositorio público de \textit{Github} (\url{https://github.com/MoonDollLuna/cidaen_m5_thesis}) bajo licencia \textit{MIT}, incluyendo tanto la memoria como el \textbf{código Python} utilizado para realizar todos los pasos del proceso de ciencia de datos.

Además, como ya se ha comentado en el Capítulo 6, la aplicación web a través de la que se ha desplegado el modelo se encuentra disponible en el enlace \url{https://cidaen-m5-thesis-lunajimenezfernandez.streamlit.app/}.


\section{Trabajo futuro}

El trabajo propuesto fue concebido como parte de una competición ya clausurada, por lo que no tiene sentido continuarlo directamente. Ahora bien, sí es posible plantear una serie de \textbf{ampliaciones al proceso realizado}, de cara a futuros trabajos de ciencia de datos:
\begin{itemize}[parsep=2pt, itemsep=2pt, topsep=4pt]
	\item \textbf{Ampliación del número de modelos:} Aun utilizando los modelos más comunes para problemas de ciencia de datos estructurados, es posible utilizar otra serie de modelos más complejos - como pueden ser las \textbf{redes neuronales} o los \textbf{\textit{stacks}} de \textit{ensembles}.
	\item \textbf{Creación de atributos:} Si bien se ha realizado un pre-procesamiento de los datos, podría haber resultado de interés añadir \textbf{atributos sintéticos} para ampliar la información útil contenida en el conjunto de datos.
	\item \textbf{Modelo disponible a través de cliente - servidor:} Actualmente, el modelo se encuentra embebido en la aplicación web - algo factible debido al tamaño ligero del modelo entrenado, pero que se haría imposible para modelos más complejos o páginas web más concurridas.
	
	Para solucionar ésto, sería razonable crear una \textbf{API (Application Web Interface)} a través de la cual se llamaría al modelo, para hospedarlo en un servidor separado independiente al funcionamiento de la propia aplicación.
\end{itemize}




% -------------------------------
% Bibliografía
% -------------------------------
\printbibliography
\addcontentsline{toc}{chapter}{Referencia bibliográfica} % Añade a la tabla de contenidos

% -------------------------------
% Apéndices
% -------------------------------

\cleardoublepage
\thispagestyle{empty}

\appendix
\chapter{Anexo 1}


\end{document}