\chapter{Contenidos del repositorio}

El repositorio en el que se almacena el trabajo realizado y la memoria contiene los siguientes recursos principales:

\begin{itemize}[leftmargin=*]
	\item \textbf{WiDS Datathon 2024 C2 - Part 1 (EDA).ipynb:} Una libreta de \textit{Jupyter} en la que se incluye el proceso de \textbf{exploración de datos}.
	\item \textbf{WiDS Datathon 2024 C2 - Part 2 (Regression Models).ipynb:} Una libreta de \textit{Jupyter} en la que se incluyen los procesos de \textbf{pre-procesamiento}, \textbf{modelado} y \textbf{evaluación} del proceso de ciencia de datos.
	\item \textbf{\\streamlit:} Una carpeta en la que se contiene todos los ficheros de código \textbf{Python} para el \textbf{despliegue} de la aplicación web a través de la que se hace disponible el modelo.
	\begin{itemize}
		\item Debido al funcionamiento de \textit{Streamlit} a la hora de desplegarse, la mayoría de recursos se encuentran duplicados en la raíz del repositorio.
	\end{itemize}
	\item  \textbf{\\thesis:} Una carpeta en la que se incluyen todos los ficheros de \textbf{LaTeX} necesarios para la compilación de la memoria.
		
		
\end{itemize}

Además, el repositorio contiene las siguientes carpetas con información adicional utilizada durante el trabajo:

\begin{itemize}[leftmargin=*]
	\item \textbf{\\data:} Una carpeta con los conjuntos de datos de la competición.
	\item \textbf{\\models:} Una carpeta con todos los modelos entrenados, tanto los 16 modelos ajustados como el modelo final seleccionado.
	\item \textbf{\\results:} Una carpeta con todos los resultados de la experimentación, tanto las estadísticas como los hiperparámetros - en formato \textit{CSV}.
\end{itemize}