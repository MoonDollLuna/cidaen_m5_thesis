\chapter{Preparación del conjunto de datos}

En este capítulo se describe el conjunto de transformaciones y técnicas aplicadas para preparar el conjunto de datos para su uso posterior durante el entrenamiento y experimentación.

En primer lugar se propone un número de \textbf{subconjuntos de atributos} de cara a reducir la dimensionalidad del conjunto de datos. Tras esto, se plantean y describen las \textbf{transformaciones} aplicadas al conjunto de datos previo al entrenamiento para estandarizar los datos y mejorar el rendimiento de los modelos.


\section{Selección de atributos}

Durante el análisis exploratorio se realizó un estudio exhaustivo de los atributos contenidos en el conjunto de datos, observándose los siguientes problemas:

\begin{itemize}
	\item \textbf{Alta dimensionalidad:} El conjunto de datos tiene \textbf{150 atributos} en total, donde la mayoría de atributos categóricos tienen \textbf{40 o más valores únicos}. Esto, unido al número de instancias bajo para dicha complejidad, puede significar que el modelo acabaría \textbf{sobreajustándose} al no poder aprender generalizaciones de forma adecuada.
	\item \textbf{Irrelevancia de los atributos:} De los 150 atributos estudiados, \textbf{la amplia mayoría no presentan correlación con la variable objetivo} - por lo que mantenerlos puede implicar una disminución del rendimiento final del modelo y un aumento del tiempo de entrenamiento.
\end{itemize}

Debido a esto, resulta necesario realizar una \textbf{selección de atributos} - proponiendo varios \textbf{subconjuntos de atributos} a utilizar durante la experimentación y selección de modelos, con el objetivo de optimizar el rendimiento del modelo reduciendo la dimensionalidad.

\subsection{Selección manual de atributos}

testing testing

\subsection{Selección automática de atributos}

\section{Preprocesamiento de los datos}

\begin{itemize}[leftmargin=*]
	\item \textbf{Atributos categóricos:}
	
	\item \textbf{Atributos numéricos:}
\end{itemize}